As you may have seen from the previous section, \egg is actually composed of several different tools, each taking care of a different step of the simulation process:
\begin{itemize}
\item \bashinline{egg-gencat}: creates a new \egg mock catalog,
\item \bashinline{egg-getsed}: pick the complete spectrum of a galaxy from an SED data base created by \bashinline{egg-gencat},
\item \bashinline{egg-2skymaker}: convert an \egg catalog into a \skymaker input catalog,
\item \bashinline{egg-gennoise}: create an empty image with noise,
\item \bashinline{egg-genmap}: paint galaxies from an \egg catalog to an empty map created with \bashinline{egg-gennoise}.
\end{itemize}

In this section, I will describe the features and capabilities of each of these programs. You can refer to this documentation if you need help understanding this or that command line argument, or if you want to discover new features that you didn't know existed. Alternatively, each program can provide you with some limited help on the spot, if you simply call
\begin{bashcode}
egg-[xxx] help
\end{bashcode}
in the terminal (where \bashinline{[xxx]} is to be replaced by the name of the program you want the help for).

\subsection{Generic information (applies to all programs)}

\subsubsection{Command line arguments}

All the tools in the \egg suite are compiled into binary executables. This means you do not need to run Python, IDL, or any other interpreter to launch them. It also means they are fast!

To specify the parameters of each programs, \egg relies exclusively on command line arguments, rather than configuration files (as done, e.g., with \tphot, \skymaker or \sextractor). The reason why is that it allows easy scripting with \bash (or your favorite shell), and also allows you to experiment and tweak the parameters directly inside the terminal rather than having to go back and forth between the terminal and the configuration file.

The syntax for command line arguments is simple: \bashinline{variable=value} or \bashinline{-variable=value} (the two are perfectly equivalent). Spaces are not allowed on either side of the \bashinline{=} sign, and if \bashinline{value} contains spaces, you must wrap it within double quotes: \bashinline{variable="some value"}. If you want to provide an array of values, the syntax is: \bashinline{variable=[value1,value2,...]}, with no space. If at least one of the values does contain a space, you must wrap the whole array within double quotes: \\
\bashinline{variable="[some value1,value2 foo,...]"}\\
(in this case, spaces can be used freely within the quotes). Lastly, if you want to provide multiple command line arguments, simply put one (or more) space between each argument: \bashinline{x1=10 x2=5}. The order of the arguments does not matter.

Regarding the \emph{type} of the argument. From the command line, everything is a string of characters. So there is no difference between \bashinline{"0.08"} and \bashinline{0.08}, or \bashinline{"foo"} and \bashinline{foo}. The values that you provide are translated into numbers/booleans/whatever by the program, using the standard C++ parsing. This means in particular that you can use scientific notation for large/small numbers (\bashinline{-1e56}). Also, if the parameter that you want to modify is a boolean (i.e., either \cppinline{0} or \cppinline{1}), you can simply write the command line argument as: \bashinline{variable} (without a value), which is equivalent to \bashinline{variable=1}.

Some examples:
\begin{bashcode}
egg-gencat area=0.08        # good
egg-gencat area = 0.08      # bad! don't put spaces around '='

egg-gencat note="my catalog v1.0" # good
egg-gencat note=my catalog v1.0   # bad! missing the quotes "..."

egg-gencat area=0.08 zmin=2 # good
egg-gencat area=0.08zmin=2  # bad! need a space between each argument

egg-gencat bands=[hst-f160w,spitzer-irac1]    # good
egg-gencat bands="[hst-f160w,spitzer-irac1]"  # good also
egg-gencat bands=[hst-f160w, spitzer-irac1]   # bad! cannot use a space in the array
egg-gencat bands="[hst-f160w, spitzer-irac1]" # good! with the quotes, it is fine
\end{bashcode}


\subsubsection{Control what is written to the standard output (terminal)}

By default, all the tools in the \egg suite run in ``silent'' mode: unless an error happens, they will not print anything in the terminal. This is mostly useful for scripted usage. However, if you want to make sure you understand what is going on, it is recommended to set the \bashinline{verbose} command line flag. It is available for all the tools, and will let each program print information about what it is doing; a short description of each step of their execution, or even progress bars for the longest steps. This is designed to alter the performances in a non-noticeable way, so do not hesitate to use it.


\subsubsection{Rules for ASCII table formatting}

Unfortunately, there are various ways an ASCII table can be defined, and building a parser to support them all is a challenging (if not impossible) task. For simplicity, the \egg tools only support one such definition. It reads as follows:
\begin{itemize}
\item The file may start with a header. It is purely descriptive and will not be read by the program. This header can span one or multiple lines, and each of these lines must start with the \cppinline{'#'} character.
\item Empty lines are allowed and ignored.
\item Each column must be separated by spaces (or tabulations) only. Properly aligning the columns is recommended for human readability, but is not mandatory.
\item Values in the table may not contain any space. Using quotes will not help you.
\item Missing values (i.e., empty ``cells'' in the table) are forbidden and cannot be understood by the program. Use a placeholder value instead (for floating point numbers, \bashinline{nan} is a good choice).
\item All floating point formats are accepted: fixed point (\cppinline{1.5}) and scientific (\cppinline{1.5e15}). The special floating point values \bashinline{+inf}, \bashinline{-inf} and \bashinline{nan} are accepted (case does not matter).
\item All the values that contain non-numeric characters that are not covered with the above rules are considered as strings.
\item A column may only contain values of a unique type: it is forbidden to mix numbers and strings.
\end{itemize}


\subsubsection{Column-oriented FITS tables}

For historical reasons tied to IDL, \egg and \phypp (the underlying C++ library) exclusively support column-oriented FITS tables. Although this is a perfectly valid way of doing things (according to the FITS standard), this is not the standard way FITS tables are used in general. It has, however, a number of advantages that I will not describe here (take a look at the \phypp documentation if you are interested). Instead, this section will tell you how you can read and write such kind of FITS tables.

{\bf In C++.} Using the \phypp library, reading and writing these tables is natural:
\begin{cppcode}
// First declare the columns you want to read (here: 1D columns of doubles)
vec1d m, z, ra, dec;

// Then read them from the file (order is irrelevant)
fits::read_table("the_file.fits", ftable(m, z, ra, dec));

// Now you can do whatever you want with these columns
m += 1+z; // some silly stuff

// Writing is as simple
fits::write_table("new_file.fits", ftable(m, z, ra, dec));
\end{cppcode}
Only the columns you need are actually read from the file, and the type of the variables you declare in C++ has to match the type of the columns that are found inside the FITS table (conversions will be performed automatically only if they imply no potential loss of data). Diagnostics will be given if this is not the case. See the \phypp documentation for more detail.

{\bf In IDL.} Column-oriented FITS tables can be read using \bashinline{mrdfits}, and written using \bashinline{mwrfits}:
\begin{idlcode}
; Read the whole table
tbl = mrdfits('the_file.fits', 1, /silent)

; Now you can do whatever you want with these columns
tbl.m += 1+tbl.z ; some silly stuff

; Writing is as simple
mwrfits, tbl, 'new_file.fits', /create
\end{idlcode}
Be careful not to forget the \idlinline{/create} keyword, else if the file already exists the function will add a new extension to the table to write the data, and this is not what you want.

{\bf In Python.} Column-oriented FITS tables are not well supported by the standard FITS I/O module from \bashinline{astropy} (you \emph{can} use \pythoninline{astropy.io.fits}, but it will be a bit cumbersome). To cope with this situation, I have written a small module that implements this in a user-friendly way; you can download it \href{https://github.com/cschreib/phypp/blob/master/python/pycolfits.py}{[here]}. Usage is as simple as above:
\begin{pythoncode}
import pycolfits

# Read the whole table
tbl = pycolfits.readfrom('the_file.fits')

# Now you can do whatever you want with these columns
tbl['m'] += 1+tbl['z'] # some silly stuff

# Writing is as simple
pycolfits.writeto('new_file.fits', tbl, clobber=True)
\end{pythoncode}
As for the IDL version, do not forget the \pythoninline{clobber=True} argument, else the function will throw an exception if the file already exists (I don't like this, but this seems to be the expected behavior in Python).

{\bf In Topcat.} If you use a language that does not support column-oriented tables, you can always try to use Topcat to open these tables and convert them into a format of your choosing (would it be ASCII or some other FITS format). Be sure to load the FITS table as ``colfit-basic''. Unfortunately it does not support all the features of column-oriented tables, so all but the simplest files will be rejected. Consider instead investing time learning one of the above languages (and maybe not IDL).
